\documentclass[a4paper,11pt]{article}

\usepackage[top=2.5cm , bottom=3cm, left = 2.5cm, right = 2.5cm]{geometry}
\usepackage{tikz}
\usetikzlibrary{shapes,arrows.meta,positioning}

\linespread{1.25}
\usepackage{enumerate, titling}
\usepackage{color,xcolor}
\usepackage[flushleft]{threeparttable}
\usepackage{amsmath, graphicx, hyperref, lipsum}
\hypersetup{
    colorlinks=true,
    linkcolor=blue,
    filecolor=magenta,      
    urlcolor=cyan}
\usepackage[authordate]{biblatex-chicago}
\addbibresource{moonopoly.bib}


\title{\vspace{-1.2cm} Moo-nopoly: Market Concentration of Veterinarians blah blah} 
\author{Emily Hanson, Vikash Kumar, Zhaoqi Li, Shuning Li, Nalinda Murray, Lance Taylor }
\date{May 10, 2024}

\begin{document}


\maketitle \vspace{-.4 in}
\begin{center}
University of Western Ontario
\rule{\textwidth}{1pt} 
\end{center}




\begin{abstract}
\lipsum[5]
\end{abstract}

\section{Introduction}
The 2022 Canadian Community Health Survey revealed that over one-third of Canadians did not receive dental care services in the year preceding the survey, despite a recommended frequency of two dental exams per year in order to maintain good oral health (Statistics Canada 2023). The survey also revealed a sizeable gap in visits to dental professionals by income: 73\% of surveyed Canadians in the highest income quintile reported having visited a dental professional in the past year, compared to only 45\% of Canadians in the lowest income quintile. Infrequent dental visits among Canadians, especially by income, may be explained in part by difficulty affording dental care. The survey suggests that approximately one in four Canadians avoid seeing dental professionals due to the cost, and this is typically worse for marginalized and immigrant groups. Additionally, more than one in three Canadians do not have any private nor public dental insurance coverage and would have to pay for services out-of-pocket.

Notably, dental services are relatively costly in Canada. Among G7 countries, Canada (414 USD) ranks second to the U.S. (518 USD) in average prices of dental services (Sankovich 2024) and is followed by the U.K. (331 USD).\footnote{These country price averages were computed across five key dental treatments: cleanings, crowns, root canals, tooth extractions, and fillings.} The level of dental care expenses is clearly an important issue for Canadians, especially in recent years, as the federal government launched the Canada Dental Benefit in December 2022 to provide coverage for children under twelve years of age whose parents do not have private insurance. This program, now called the Canadian Dental Care Plan (CDCP), has since been expanded to provide coverage to uninsured Canadians under eighteen years of age, those with disabilities, and senior citizens satisfying a specific income criteria. The final expansion to working-age Canadians satisfying a specific income criteria is expected to take place by 2025. Providing insurance coverage for a quarter of Canadians, the plan is expected to cost approximately 4.4 billion CAD per year (Rachini 2023).

Despite the recent roll-out of the CDCP to senior Canadians, participation of dentists in the program is currently low, which may be due to the fact that the program requires participating dental clinics to provide services at lower rates for those covered by the CDCP (von Stackelberg 2024). The apparent reluctance of dentists to participate in the CDCP may be particularly worrisome for Canadians living in more isolated towns, who may lack dental care options and may have to travel far distances to find a participating dentist. To shed light on this issue, this paper explores market concentration and competition in the Canadian dental industry. This may speak to whether the CDCP can be successful; that is, whether competitive pressures are sufficient to induce participation among dentists. Importantly, although an analysis of pricing and competition is out of the scope of this study, market concentration and competition in the Canadian dental industry may provide insights into why the prices of dental services in Canada may be as high as they are in the first place, necessitating the recent policy changes.

To do so, we replicate the analysis of market concentration and entry of Bresnahan and Reiss (1991) for the Canadian dental industry in Canada. In particular, we . . .

Summarize findings.

The remainder of this paper is as follows. Section 2 describes the Canadian dental industry and the data used in our analysis. Section 3 describes the model and presents the results from estimating the model using our Canadian data. Section 4 concludes.

\section{Industry and Data Description}

\section{Model and Results}


\begin{table}[h]  
\begin{threeparttable}
\caption{Market and Businesses Counts} % title name of the table  
\centering % centering table  
\begin{tabular}{l c c } % creating 10 columns  
\hline\hline   
 Geography & Population Centres & Veterinarian Businesses   
\\ [0.5ex]  
\hline   
% Entering 1st row  
Canadian Provinces & 1014 & 1929 \\
Within CSA or surrounding 20km & 737 & 1573 \\
Within 10km of other Population Centre & 42 & 16 \\
Remaining Population Centres & 235 & 114 \\

Remaining PCs and surrounding 1km* & 235 & 186 \\

Remaining PCs and surrounding 2km  & 235 & 200 \\

Remaining PCs and surrounding 5km  & 235 & 209 \\

% [1ex] adds vertical space  
\hline % inserts single-line  
\end{tabular}  
\begin{tablenotes}
    \small  *This set of population centres including 186 veterinarian businesses was chosen for analysis. 
\end{tablenotes}
\end{threeparttable}
\end{table}  

Model and Results: ????

\section{Conclusion}


\newpage
\printbibliography

\newpage
\section*{Appendix}
\subsection*{Data Collection Process}

\subsection*{Notes for Future Selves}
\begin{itemize}
    \item 
\end{itemize}




\end{document}
